\documentclass{IEEEtran}
\usepackage[T1]{fontenc}
\usepackage{biblatex}
\usepackage{listings}
\addbibresource{Expose.bib}

\begin{document}
\title{Vergleich von Methoden zur Lösung des Capacitated Vehicle Routing Problems (CVRP)}

\author{\IEEEauthorblockN{Walter Hoos, Alexander Hübler, David Winzer}
\IEEEauthorblockA{\\Department Sichere Informationssysteme,
FH Oberösterreich, Campus Hagenberg}}

\maketitle

\begin{abstract}
    Das Capacitated Vehicle Routing Problem (CVRP) ist ein kombinatorisches 
    Optimierungsproblem mit vielzähligen praktischen Anwendungen 
    in der Logistik und Tourenplanung. Ziel ist es, eine kostenminimale Reihe von 
    Routen für eine Flotte von Lieferfahrzeugen mit begrenzter Kapazität zu finden, 
    sodass alle Kundennachfragen bedient werden. Das CVRP ist NP-komplex und gehört 
    zu den am intensivsten erforschten Problemen im Bereich der Tourenplanung, wodurch
    es eine Vielzahl von Lösungsansätzen gibt. Einige von ihnen sollen in dieser Arbeit
    miteinander verglichen und evaluiert werden.
\end{abstract}

\section{Beschreibung des Problems}
    Das Vehicle Routing Problem (VRP), auch bekannt als das Standardproblem der 
    Tourenplanung, repräsentiert ein zentrales kombinatorisches Optimierungsproblem 
    innerhalb der logistischen Planung und des Operations Research. Die grundlegende 
    Zielsetzung des VRP besteht darin, für eine Flotte von Fahrzeugen
    die optimale Routenplanung zu ermitteln, um eine Menge von Kundenstandorten unter 
    Berucksichtigung bestimmter Restriktionen zu beliefern. Diese Restriktionen 
    können unter anderem die Kapazität der Fahrzeuge, Zeitfenster für die Lieferungen 
    und die maximale Dauer der Touren umfassen. \newline\newline
    Konzeptuell stellt das VRP eine Erweiterung des Traveling Salesman Problem (TSP) 
    dar. Wahrend das TSP die Suche nach der kürzesten, möglichen Route für eine/n 
    einzelne/n Verkäufer*in, der jede Stadt genau einmal besucht und zum Ausgangspunkt 
    zurückkehrt, zum Gegenstand hat, erweitert das VRP diese Problemstellung auf mehrere 
    Fahrzeuge und berücksichtigt zusätzlich eine Reihe von praxisrelevanten 
    Nebenbedingungen. Das VRP tragt somit einer komplexeren und realitätsnäheren 
    Anforderung der Tourenplanung Rechnung. Beschrieben wurde dieses Problem erstmals 
    durch G. B. Dantzig und J. H. Ramser im Jahr 1959. \newline\newline
    Im Zuge dieser Arbeit soll eine spezifische Instanz des CVRP welche von 
    P. Augerat\cite{augerat} beschrieben wurde genutzt werden, um die Effektivität 
    mehrerer Methoden zur Lösung des Problems zu evaluieren. 

    \subsection{A-n80-k10}
    Bei der Probleminstanz handelt es sich konkret um das Problem A-n80-k10 aus dem Set A\footnote{\url{http://www.vrp-rep.org/datasets/download/augerat-1995-set-a.zip}}. 
    

\section{Beschreibung der verwendeten Werkzeuge (Methodik)}
    Zur Lösung des CVRP gibt es verschiedene Methoden und Verfahren. 
    Einige der bekanntesten Methoden gehören zur Kategorie der heuristischen
    Algorithmen, wobei diese sich in zwei Hauptgruppen
    unterteilen lassen:\newline\newline
    1. Konstruktive Heuristiken: Diese Algorithmen erstellen iterativ eine zulässige 
    Lösung, indem sie schrittweise Kunden zu Routen hinzufügen. Gängige 
    Konstruktionsheuristiken für das CVRP sind beispielsweise die Savings-Heuristik, 
    die Sweep-Heuristik und die Cluster-First, Route-Second-Methode. \newline\newline
    2. Verbesserungsheuristiken: Diese Heuristiken nehmen eine bereits existierende 
    zulässige Lösung und versuchen, diese iterativ durch lokale Suchoperationen zu 
    verbessern. Bekannte Verbesserungsheuristiken für das CVRP sind unter anderem die 
    $\lambda$-Interchange-Heuristik, die Or-Opt-Heuristik und die Tabu-Suche.
\subsection{Offspring Selection Genetic Algorithm (OSGA)}


\subsection{Google OR-Tools}
Die Google OR-Tools sind eine Open-Source-Software-Suite für Operations 
Research, die von Google entwickelt und bereitgestellt wird. Sie bietet eine Reihe von 
Algorithmen und Modellierungshilfen für verschiedene Optimierungsprobleme, darunter 
auch Lösungsansätze für das Capacitated Vehicle Routing Problem (CVRP). \newline\newline
Die OR-Tools sind in mehreren Programmiersprachen wie C++, Python und Java verfügbar und 
ermöglichen so eine unkomplizierte Integration in bestehende Softwaresysteme. Zudem bieten 
sie leistungsfähige Visualisierungsmöglichkeiten zur Darstellung der berechneten Routen.\newline\newline
Die Implementierung des CVRP in den Google OR-Tools erfolgt durch die Verwendung des 
integrierten Routing-Moduls. Dieses Modul bietet verschiedene Algorithmen und Heuristiken, 
die speziell für Routing-Probleme entwickelt wurden.

\subsection{Programming Z3}
Der Z3 Solver ist ein leistungsfähiges Satisfiability-Modulo-Theories (SMT) Werkzeug, 
das von Microsoft Research entwickelt wurde. Es findet breite Anwendung in verschiedenen 
Bereichen, wie der formalen Verifikation, der symbolischen Ausführung und der 
Constraint-Programmierung. Im Kontext des Capacitated Vehicle Routing Problems (CVRP) wird
der Z3 Solver als Optimierungsmodul eingesetzt, um effiziente Lösungen für dieses Problem
zu finden.

\section{Ergebnisse}


\section{Conclusio}
% Conclusion content goes here

\printbibliography


\appendix
\section{A-n80-k10}
\lstinputlisting{../Problem/problem.py}

\section{Code}
\subsection{Z3}
\subsection{Or-Tools}

\end{document}