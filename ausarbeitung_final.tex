\documentclass{IEEEtran}
\usepackage[T1]{fontenc}
\usepackage{biblatex}
\usepackage{listings}
\usepackage{tabularx}
\usepackage{capt-of}
\usepackage{cuted}
\usepackage{graphicx}

\addbibresource{Expose.bib}

\begin{document}
\title{Vergleich von Methoden zur Lösung des Capacitated Vehicle Routing Problems (CVRP)}

\author{\IEEEauthorblockN{Walter Hoos, Alexander Hübler, David Winzer}
\IEEEauthorblockA{\\Department Sichere Informationssysteme,
FH Oberösterreich, Campus Hagenberg}}

\maketitle

\begin{abstract}
    Das Capacitated Vehicle Routing Problem (CVRP) ist ein kombinatorisches 
    Optimierungsproblem mit vielzähligen praktischen Anwendungen 
    in der Logistik und Tourenplanung. Ziel ist es, eine kostenminimale Reihe von 
    Routen für eine Flotte von Lieferfahrzeugen mit begrenzter Kapazität zu finden, 
    sodass alle Kundennachfragen bedient werden. Das CVRP ist NP-komplex und gehört 
    zu den am intensivsten erforschten Problemen im Bereich der Tourenplanung, wodurch
    es eine Vielzahl von Lösungsansätzen gibt. Einige von ihnen sollen in dieser Arbeit
    miteinander verglichen und evaluiert werden.
\end{abstract}

\section{Beschreibung des Problems}
    Das Vehicle Routing Problem (VRP), auch bekannt als das Standardproblem der 
    Tourenplanung, repräsentiert ein zentrales kombinatorisches Optimierungsproblem 
    innerhalb der logistischen Planung und des Operations Research. Die grundlegende 
    Zielsetzung des VRP besteht darin, für eine Flotte von Fahrzeugen
    die optimale Routenplanung zu ermitteln, um eine Menge von Kundenstandorten unter 
    Berucksichtigung bestimmter Restriktionen zu beliefern. Diese Restriktionen 
    können unter anderem die Kapazität der Fahrzeuge, Zeitfenster für die Lieferungen 
    und die maximale Dauer der Touren umfassen. \newline\newline
    Konzeptuell stellt das VRP eine Erweiterung des Traveling Salesman Problem (TSP) 
    dar. Wahrend das TSP die Suche nach der kürzesten, möglichen Route für eine/n 
    einzelne/n Verkäufer*in, der jede Stadt genau einmal besucht und zum Ausgangspunkt 
    zurückkehrt, zum Gegenstand hat, erweitert das VRP diese Problemstellung auf mehrere 
    Fahrzeuge und berücksichtigt zusätzlich eine Reihe von praxisrelevanten 
    Nebenbedingungen. Das VRP tragt somit einer komplexeren und realitätsnäheren 
    Anforderung der Tourenplanung Rechnung. Beschrieben wurde dieses Problem erstmals 
    durch G. B. Dantzig und J. H. Ramser im Jahr 1959. \newline\newline
    Im Zuge dieser Arbeit soll eine spezifische Instanz des CVRP welche von 
    P. Augerat\cite{augerat} beschrieben wurde genutzt werden, um die Effektivität 
    mehrerer Methoden zur Lösung des Problems zu evaluieren. 

    \subsection{A-n80-k10}
    Bei der Probleminstanz handelt es sich konkret um das Problem A-n80-k10 aus dem Set A\footnote{\url{http://www.vrp-rep.org/datasets/download/augerat-1995-set-a.zip}}. 

    \subsection{Formale Beschreibung des Problems}
    Die formale Beschreibung des Problems wurde mithilfe von \cite{tothVigo2002} durchgeführt.

    Der Lösungsraum besteht aus $K$ Kanten zwischen $S$ Standorten. Diese Standorte haben jeweils einen Bedarf $B$.
    Eine Strecke zwischen zwei Standorten kann durch ihre zwei Standorte $(a,b)$ ausgedrück werden.
    Es existieren Fahrzeuge $f$ aus der Menge aller Fahrzeuge $F$, welche jeweils eine Maximalladung $M$ mit sich führen können.
    Zusätzlich muss eine Binärvariable $x_{abf}$ eingeführt werden, welche besagt, ob ein Fahrzeug die Strecke $(a,b)$ nutzt ($f=1$) oder nicht ($f=0$).
    Zudem besagt $y_{af}$ ob ein Fahrzeug $f$ den Standort $a$ besucht ($f=1$) oder nicht ($f=0$).
    Die Kosten der Strecke $(a,b)$ wird als $k_{ab}$ beschrieben.
    Es gilt die Zielfunktion $Z$ zu minimieren: 
    \[Z = \sum_{(a,b)\in S} \sum _{f=1} ^F x_{abf} k_{ab}\]

    Zudem gelten folgende Nebenbedingungen:
    \begin{itemize}
        \item Jeder Standort $S$ darf nur einmal besucht werden ausgenommen das Depot $s_0$ :
        \[ \sum_{f=1} ^F y_{a f} =1 \forall a \in S\setminus (s_0) \]

        \item Jede Route muss am Depot $s_0$ starten:
        \[\sum_{f=1}^{F} y_{s_0f} =1\]

        \item Ein Standort darf nur genau dann besucht werden wenn $y_{abf} = 1$:
        \[ \sum_{b}^{S} x_{abf} \iff y_{af}=1 \forall (a,b) \in S \]

        \item Die Maximalladung darf in keinem der Fahrzeuge überschritten werden:
        \[\sum_{a=1}^{S} B_{a}y_{af} \leq M \forall f \in S \]
    \end{itemize}


\section{Beschreibung der verwendeten Werkzeuge (Methodik)}
    Zur Lösung des CVRP gibt es verschiedene Methoden und Verfahren. 
    Einige der bekanntesten Methoden gehören zur Kategorie der heuristischen
    Algorithmen, wobei diese sich in zwei Hauptgruppen
    unterteilen lassen:\newline\newline
    1. Konstruktive Heuristiken: Diese Algorithmen erstellen iterativ eine zulässige 
    Lösung, indem sie schrittweise Kunden zu Routen hinzufügen. Gängige 
    Konstruktionsheuristiken für das CVRP sind beispielsweise die Savings-Heuristik, 
    die Sweep-Heuristik und die Cluster-First, Route-Second-Methode. \newline\newline
    2. Verbesserungsheuristiken: Diese Heuristiken nehmen eine bereits existierende 
    zulässige Lösung und versuchen, diese iterativ durch lokale Suchoperationen zu 
    verbessern. Bekannte Verbesserungsheuristiken für das CVRP sind unter anderem die 
    $\lambda$-Interchange-Heuristik, die Or-Opt-Heuristik und die Tabu-Suche.

\subsection{Offspring Selection Genetic Algorithm (OSGA)}
Genetische Algorithmen sind eine Form von metaheuristischen Verfahren, welche die natürliche Selektion in der Genetik nachahmen. 
Hierbei werden Individuen nach dem prinzip "surival of the fittest" ausgewählt und so soll über mehrere Generationen hinweg so eine nahezu optimale Lösung generiert werden.
Die Grundsätzliche Funktionsweise von GA ist wiefolgt\cite{Winkler2024}:
\begin{enumerate}
    \item Initialisierung einer Population von Chromosomen
    \item Fitness evaluation der Chromosomen
    \item Selektion der Eltern
    \item Variation durch Crossover und Mutation
    \item Fitness evaluation
    \item (Überlebensselektion)
    \item Wenn Abbruchkriterien erfüllt sind Ende, ansonsten Sprung zu 3.
\end{enumerate}

Traditionelle GA haben einen primären Selektionsprozess zum Finden der Eltern. In OSGA existiert neben diesem ein weiterer Selektionsprozess, 
welcher die Auswahl der Nachkommen steuert\cite{affenzeller-wagner}. Die genutzte OSGA instanz wurde mittels der Software Heuristiclab \footnote{\url{https://dev.heuristiclab.com/trac.fcgi/}} erzeugt. 
Die genutzten Parameter können in Tabelle \ref{OSGA_table} gefunden werden. 
% \begin{table}
%    \renewcommand{\arraystretch}{1.3}
%    \caption{Genutzte Parameter für OSGA}
%    \label{OSGA_table}
%    \centering
%    \begin{tabular}{c||c||c}
%    \hline
%    \bfseries  Parameter & \bfseries  Wert & \bfseries Beschreibung\\
%    \hline\hline\hline

%    Elites                     & 5                                  & 5 \\
%    Crossover                  & Alba Permutation Crossover         & 5 \\
%    Maximum Generations        & 10000                              & 5 \\
%    Maximum Selection Pressure & 100                                & 5 \\
%    Mutation Probability       & 10\%                               & 5 \\
%    Mutator                    & AlbaCustomerInsertionManipulator   & 5 \\
%    Population Size            & 500                                & 5 \\
%    Selected Parents           & 300                                & 5 \\
%    Selektor                   & Tournament Selektor (Group Size 2) & 5 \\
%    \hline
%    \end{tabular}
% \end{table}

\begin{table}
    \renewcommand{\arraystretch}{1.3}
    \caption{Genutzte Parameter für OSGA}
    \label{OSGA_table}
    \centering
    \begin{tabular}{c||c}
    \hline
    \bfseries  Parameter & \bfseries  Wert \\
    \hline\hline\hline
 
    Elites                     & 5                                  \\
    Crossover                  & Alba Permutation Crossover         \\
    Maximum Generations        & 10000                              \\
    Maximum Selection Pressure & 100                                \\
    Mutation Probability       & 10\%                               \\
    Mutator                    & AlbaCustomerInsertionManipulator   \\
    Population Size            & 500                                \\
    Selected Parents           & 300                                \\
    Selektor                   & Tournament Selektor (Group Size 2) \\
    \hline
    \end{tabular}
 \end{table}

OSGA wurde in dieser Konfiguration 20 mal ausgeführt und eine Evaluierung der Ergebnisse erstellt.

\begin{figure}
    \centering
    \includegraphics[width=2.5in]{../OSGA/Routen.png}
    \caption{Beste durch OSGA gefundene Routen}
    \label{paths}
\end{figure}

\begin{figure}
    \centering
    \includegraphics[width=2.5in]{../OSGA/Streuung.png}
    \caption{Streuung der durch OSGA erzielten Ergebnisse}
    \label{box}
\end{figure}

In dieser Konfiguration Liefert OSGA im besten Fall Ergebnisse mit einer Routenlänge von 3339m bis zu. 
Die Streuung der Ergebnisse ist nicht besonders groß jedoch liegt das best erzielte Ergebnis weit von dem bekannten Optimum.
Es ist durchaus möglich, dass durch eine Modifikation einiger Parameter das Ergebnis noch verbessert werden könnte, jedoch ist vermutlich eine andere Metaheurisik besser geeignet um ein gutes Ergebnis zu finden.
Ein möglicher Algorithmen hierfür wären besispielsweise Simulated Annealing.

\subsection{Google OR-Tools}
Die Google OR-Tools \cite{GoogleORTools} sind eine Open-Source-Software-Suite für Operations 
Research (OR), die von Google entwickelt und bereitgestellt wird. Sie bietet eine Reihe von 
Algorithmen und Modellierungshilfen für verschiedene Optimierungsprobleme, darunter 
auch Lösungsansätze für das Capacitated Vehicle Routing Problem (CVRP). \newline\newline
Die OR-Tools sind in mehreren Programmiersprachen wie C++, Python und Java verfügbar und 
ermöglichen so eine unkomplizierte Integration in bestehende Softwaresysteme. Zudem bieten 
sie leistungsfähige Visualisierungsmöglichkeiten zur Darstellung der berechneten Routen.\newline\newline
Unsere Implementierung des CVRP in den Google OR-Tools erfolgt durch die Verwendung des 
Routing-Moduls \cite{GoogleCVRP} in Python. Dieses Modul bietet verschiedene Algorithmen und Heuristiken, 
die speziell für Routing-Probleme entwickelt wurden, sowie weitere Einstellmöglichkeiten
für den Routing-Solver wie zum Beispiel: 
\begin{itemize}
    \item{solution\_limit: Maximale Anzahl von Lösungen, die der Solver generieren soll.}
    \item{time\_limit.seconds: Maximale Zeit in Sekunden, die der Solver für die Lösung des Problems verwenden soll.}
    \item{lns\_time\_limit.seconds: Maximale Zeit in Sekunden, die der Solver für die Lösung des Problems mit Large Neighborhood Search (LNS) verwenden soll.}
\end{itemize}
Außerdem stehen verschiedene Strategien zur Auswahl, um eine initiale Lösung zu finden wie z.B. 
kostengünstigste Verbindung, Savings-Algorithmus, Sweep-Algorithmus oder eine Variante des 
Christofides-Algorithmus. In unserer Implementierung verwenden wir, wie von Google empfohlen \cite{GoogleCVRP},
die Strategie der kostengünstigsten Verbindung. \newline\newline
Für die lokale Suche zur Verbesserung der initialen Lösung gibt es ebenfalls mehrere metaheuristische 
Ansätze wie GreedyDescent, GuidedLocalSearch, SimulatedAnnealing oder TabuSearch. Von Google empfohlen
und daher von uns verwendet wird die GuidedLocalSearch-Heuristik. \newline\newline
Weitere Parameter erlauben die Steuerung der Propagierung von Constraints und entscheiden, 
ob vollständige oder nur eingeschränkte Propagierung verwendet wird.


\subsection{Z3 SMT Solver}

Der Z3 Solver ist ein leistungsfähiges Satisfiability-Modulo-Theories (SMT) Werkzeug, 
das von Microsoft Research entwickelt wurde. Es findet breite Anwendung in verschiedenen 
Bereichen, wie der formalen Verifikation, der symbolischen Ausführung und der 
Constraint-Programmierung. Im Kontext des Capacitated Vehicle Routing Problems (CVRP) wird
der Z3 Solver als Optimierungsmodul eingesetzt, um effiziente Lösungen für dieses Problem
zu finden.

Bei dem CVRP handelt es sich um ein NP-Komplettes Problem, da es zur Klasse NP zählt (nichtdeterministisch polynomiell zeitlösbar). Ein Problem wird als NP-komplett klassifiziert, wenn es zwei Hauptkriterien erfüllt: 

\begin{itemize}
    \item Erstens, jede vorgeschlagene Lösung des Problems kann in polynomieller Zeit verifiziert, jedoch nur in Exponentieller oder Faktorieller Zeit berechnet werden.
    \item Es kann jedes andere Problem in NP kann in polynomieller Zeit auf dieses Problem reduziert werden. Dies wird üblicherweise durchgeführt, indem ein bekanntes NP-komplettes Problem auf das zu untersuchende Problem reduziert wird.
\end{itemize}

Da das CVRP eine Verallgemeinerung des TSP ist und das TSP bekanntlich NP-komplett ist, gilt das CVRP ebenfalls als NP-komplett. Dies bedeutet, dass es unwahrscheinlich ist, dass ein Algorithmus existiert, der alle CVRP-Instanzen in polynomieller Zeit lösen kann. 

Der Z3-Solver kann prinzipiell auch für NP-komplette Probleme wie das CVRP  eingesetzt werden. Allerdings bedeutet die NP-Komplettheit, dass die erwartete Rechenzeit zur Lösung des Problems exponentiell mit der Größe der Eingabe wachsen kann. Obwohl Z3 fortschrittliche Heuristiken und Optimierungen verwendet, bleibt die grundsätzliche Herausforderung, dass die Lösung von NP-vollständigen Problemen in der Regel exponentielle Rechenzeit erfordert. Bei großen Instanzen kann dies dazu führen, dass die Rechenzeit für die praktische Problemösung untragbar wird.

In der Praxis wird oft auf heuristische oder approximative Algorithmen zurückgegrifen, um große Instanzen von NP-vollständigen Problemen wie dem CVRP zu lösen. Diese Ansätze sind zwar meistens nicht die optimale Lösung, aber sie liefern oft ausreichend gute Ergebnisse in einer praktikablen Zeit.

Es wurde eine beispielhafte Implementierung einer Lösung für das CVRP mithilfe des Z3-Solvers implementiert \footnote{\url{https://github.com/alx-hblr/FMT2-CVRP-eval/tree/main}}. Die beispielhaften Daten mit der Problemgröße von 6 Halstestellen und 2 Fahrzeugen (Problemgröße 12) kann inenrhalb kürzester Zeit (weniger als 1 Sekunde) gelöst werden. Ein CVRP mit der Problemgröße (Anzahl von Haltestellen multipliziert mit der Anzahl der Fahrzeuge) von 30 konnte in circa einer Sekunde gelöst werden. Die Problemgröße von 36 benötigt bereits eine Durchlaufzeit von 10 Sekunden. Bei einem beispielhaften Problem mit 16 Haltestellen und 3 Fahrzeugen, also einer Problemgröße von 48 beträgt die Durchalufzeit bereits mehrere Minuten. 

Daher konnte das kleinste Problem von Augerat (32 Haltestellen mit 5 Fahrzeugen, also einer Problemgröße von 160) nach einer Rechenzeit von mehreren Stunden nicht durch die Z3-Optimierungsroutinen gelöst werden. Daher würde das untersuchte Problem mit 80 Halstestellen und 10 Fahrzeugen (Problemgröße von 800) exponentiell länger dauern und kann nicht in einer realistischen Zeit gelöst werden.


\section{Conclusio}
Aufgrund der Eigenschaft von NP-vollständigen Problemen kann nicht  sicher festgelegtwerden, was die absolut beste Lösung ist., da die gewählten heuristischen Lösungsansätze zwar meist eine gute, jedoch niemals eine huntertprozentige beste Lösung für das gegebene Problem sicherstellen können. 
Für die gewählte CVRP Instanz liegt die beste bekannte Lösung bei einer Distanz von 1763m\cite{Heuristiclab}. Diesem Ergebnis am nähesten kommt die Implementierung von Google durch die Google OR-Tools, welche eine Lösung mit einer Routenlänge von 1785m liefert, bei einer vordefinierten Durchalufzeit von 120 Sekunden und den oben beschriebenen, von Google empfohlenen Parametern. Diese Lösung ist schon sehr nahe an der besten bekannten Lösung von 1763m. Der Offspring Selection Genetic Algorithm (OSGA) lieferte im Vergleich dazu ein bestes Ergebnis von 3339m und der Z3 Solver konnte das Problem aufgrund der Größe nicht lösen. 

Durch den Z3 Solver konnte leider keine Lösung für das gewählte CVRP erreicht werden, da es sich beim CVRP um ein NP-Vollständiges Problem handelt und der Solver daher durch größere Problemgrößen exponentiell längere Durchlaufzeiten benötigt. Daher konnte die Funktionalität des Solvers selbst durch kleinere Probleme zwar bewiesen, das gewählte Problem mit der Größe von 80 Haltestellen udn 10 Fahrzeugen jedoch nicht in einer zumutbaren Zeit gelöst werden. 

\printbibliography
\newpage
\appendix
\end{document}